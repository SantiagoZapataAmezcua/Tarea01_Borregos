\documentclass[12pt]{report}
\usepackage[utf8]{inputenc}
\usepackage[T1]{fontenc}
\usepackage[spanish,es-noshorthands]{babel}
\usepackage[margin=1in]{geometry}
\usepackage{array}
\usepackage{amsmath, amssymb, mathrsfs}
\usepackage{fdsymbol}
\usepackage{forest}
\usepackage{prooftrees}
\usepackage{tabularx}
\usepackage{graphicx}
\usepackage{float}
\usepackage{booktabs}
\usepackage{marginnote}
\usepackage{tcolorbox}
\usepackage{tikz}
\usetikzlibrary{shapes.geometric, arrows}
\usepackage{xcolor}
\usepackage{enumitem}
\usepackage{hyperref}
\usepackage{ragged2e}

\definecolor{azul}{RGB}{25,94,164}
\setlist[itemize]{leftmargin=*, topsep=4pt}
\setlist[enumerate]{leftmargin=*, topsep=6pt}

\begin{document}

% ------------------ PORTADA ------------------
\pagenumbering{gobble}
\begin{titlepage}
\centering
{\bfseries\LARGE Universidad Nacional Autonoma de M\'exico \par}
\vspace{1cm}
{\scshape\Large Facultad de Ciencias \par}
\vspace{3cm}
{\scshape\Huge Tarea 1  \par}
\vspace{3cm}
{\itshape\Large Fundamentos de Bases de Datos \par}
\vfill
{\scshape\Huge Equipo ``BORREGOS'' \par}
\vspace{3cm}
{\Large Aron\\ Rocks\\ Isaac Giovani Escobar Gonzalez \\ Kevin\\ Santiago Zapata Amezcua \par}
\vfill
\end{titlepage}
\clearpage

% ------------------ CONTENIDO ------------------
\pagenumbering{arabic}
\setcounter{page}{1}

\section*{Tarea 1: Conceptos b\'asicos}

\begin{enumerate}[label=\textbf{\arabic*.}, leftmargin=*]

\item \textbf{Conceptos generales:}

\begin{enumerate}[label=\textbf{\alph*.}, leftmargin=*, itemsep=1.0em]

\item Explica las caracter\'isticas fundamentales del enfoque de bases de datos frente al uso de hojas de c\'alculo. \\
\textit{¿Qu\'e limitaciones enfrentan las hojas de c\'alculo cuando se trata de escalabilidad, integridad y consistencia de datos?} \\
Describe al menos dos escenarios concretos:
\begin{itemize}
  \item Uno donde no sea adecuado usar hojas de c\'alculo.
  \item Otro donde s\'i sea preferible optar por ellas. Justifica cada caso.
\end{itemize}

    \textbf{Respuesta:}\\
\textbf{Caracterisicas:}
\begin{itemize}
    \item \textbf{A futuro(escalable):} 
    \begin{itemize}
        \item Las bases de datos soportan millones de registros y múltiples usuarios accediendo al mismo tiempo sin pérdida de rendimiento.
        \item Las hojas de cálculo se vuelven lentas e inestables con grandes volúmenes ademas de que Excel tiene un máximo de 1.048.576 filas y 16.384 columnas.
    \end{itemize}
    
    \item \textbf{Integridad y consistencia:} 
    \begin{itemize}
        \item Las bases de datos aplican reglas de integridad (claves, restricciones) que evitan duplicados o inconsistencias en los datos.
        \item Las hojas de cálculo dependen de que el usuario mantenga fórmulas y datos correctos, lo que puede provocar errores humanos.
    \end{itemize}

    \item \textbf{Colaboración:}
    \begin{itemize}
        \item Las bases de datos permiten acceso concurrente con control de transacciones.
        \item Las hojas de cálculo tienen riesgos de sobrescritura o conflictos de edición.
    \end{itemize}

    \item \textbf{Seguridad:}
    \begin{itemize}
        \item Las bases de datos gestionan usuarios, roles y permisos usando a la vez las keys que permiten el acceso a estos datos.
        \item Las hojas de cálculo solo ofrecen protecciones básicas de celdas o contraseñas, pero no asegura la integridad de los datos.
    \end{itemize}
\end{itemize}

\textbf{Limitaciones de las hojas de cálculo}
 
Las hojas de cálculo presentan limitaciones en términos de:
\begin{itemize}
    \item Escalabilidad: no soportan grandes volúmenes de datos ni múltiples usuarios simultáneos sin perder rendimiento ademas que no son finitas las columnas y filas, por lo que no pueden abarcar una cantidad infinita de datos sin antes acabarse.
    \item Integridad y consistencia: dependen de la gestión manual del usuario, aumentando el riesgo de errores y datos inconsistentes, por lo que no mantiene la integridad de los datos.
\end{itemize}
\newpage
\textbf{Escenarios concretos}\\
\textbf{-Escenario donde NO es adecuado usar hojas de cálculo}
\begin{itemize}
    \item La empresa Amazon con mas de 50,000 productos, ademas de tener ventas en tiempo real y múltiples sucursales donde almacenan los productos para su repartición.
    \item \textbf{Justificación:} Las hojas de cálculo no pueden manejar tantas transacciones simultáneas ni garantizar integridad, ademas de que debe actualizarse manualmente cada vez que se hace una compra y venta. Mientras que una base de datos permite controlar mercancías almacenadas por una empresa de manera mas eficiente, ademas de registrar ventas en línea y evitar la venta de productos inexistentes.
\end{itemize}

\textbf{-Escenario donde SÍ es preferible usar hojas de cálculo}
\begin{itemize}
    \item En un equipo universitario en la facultad de Ciencias de la UNAM que gestiona una lista de 50-200 participantes de inscripciones a las carreras y genera reportes sencillos de cada alumno y su información personal.
    \item \textbf{Justificación:} El volumen de datos es bajo, no se requiere acceso concurrente complejo y la facilidad de uso de Excel permite rapidez sin necesidad de instalar un SMBD, ademas de una facilidad de emplear una hoja de calculo que una base de datos sin conocimientos previos.
\end{itemize}

\item Sup\'on que una empresa en crecimiento necesita implementar un Sistema Manejador de Bases de Datos (SMBD) para centralizar su informaci\'on operativa (ventas, clientes, inventario y log\'istica). \\
Elige dos posibles soluciones (p.\,e., PostgreSQL, Oracle, MySQL, SQL Server, etc.) y compara sus ventajas y desventajas en t\'erminos de: rendimiento ante grandes vol\'umenes de datos; escalabilidad a futuro; costos de licenciamiento y mantenimiento; soporte t\'ecnico y comunidad. Con base en tu an\'alisis, argumenta cu\'al soluci\'on recomendar\'ias y por qu\'e.\\

\textbf{Respuesta:}

\textbf{PostgreSQL}
\begin{itemize}
    \item \textbf{Ventajas:} PostgreSQL tiene una gran capacidad para manejar consultas complejas,también una excelente escalabilidad,junto con un soporte avanzado para tipos de datos como los JSON, ademas de ser código abierto y tener una comunidad activa donde se pueden reportar y solucionar errores de manera mas practica y optimiza ble.
    \item \textbf{Desventajas:} Lamentablemente tiene una curva de aprendizaje más pronunciada y configuración inicial más exigente,  por lo que acostumbrarse a su uso puede ser demasiado tedioso y confuso.Ademas de tener un rendimiento relativamente más lento en bases de datos pequeñas,junto con una  falta de soporte oficial centralizado en comparación con otros sistemas para un mejor manejo.
\end{itemize}
\newpage
\textbf{MySQL}
\begin{itemize}
    \item \textbf{Ventajas:} MySQL al ser un sistema que se basa en un código abierto, le permite a desarrolladores y pequeñas empresas contar con una solución estandarizada y para sus aplicaciones.
    Cambien este sistema permite realizar una gestión de los datos de una forma organizada y ordenada.
    Otra de sus ventajas es que lo pueden utilizar varias personas a la vez y efectuar varias consultas al mismo tiempo, lo que lo hace que sea muy versátil.Y lo que lo hace mejor es que es fácil de instalar y configurar
    \item \textbf{Desventajas:} Sin embargo tiene un menor rendimiento en consultas analíticas complejas, ademas de que no es el más amigable con los los programas que actualmente se utiliza, y agregándole que  tiene ciertas limitaciones en funciones avanzadas junto con altos costos en la versión empresarial.
\end{itemize}

\textbf{Recomendación}
Por lo que para una empresa en crecimiento que busca centralizar operaciones de ventas, clientes, inventario y logística, la mejor opción es \textbf{PostgreSQL}, ya que ofrece mayor capacidad de escalabilidad, solidez en el manejo de datos complejos y ausencia de costos de versiones empresariales o licencias. Por que hay que tener en cuenta que MySQL es adecuado para proyectos pequeños o con requerimientos menos complejos, pero PostgreSQL proporciona una base más robusta y preparada para el crecimiento a largo plazo.






\item Una empresa de log\'istica necesita adaptar su Sistema de Base de Datos para incluir nuevas funcionalidades, como el seguimiento en tiempo real de paquetes y una nueva categor\'ia de clientes corporativos. Antes de realizar estos cambios, el equipo de desarrollo debe considerar los niveles de independencia de los datos. Responde lo siguiente:
\begin{itemize}
  \item ¿Qu\'e riesgos podr\'ian surgir s\'i no existe independencia l\'ogica entre el dise\~no conceptual y las aplicaciones que consumen los datos?
  
        La independencia logica nos asegura el cambio en nuestro modelo de datos sin romper las aplicaciones, sin esta no existe:
        Los cambios pequeños rompen cosas por ejemplo el renombrar entidades o el separar direcciones en distintas columnas, las consultas y los reportes que usan estos nombres fallan.
        De igualmanera sin independencia logica implica actualizar cada microservicio para volver a dar relacion y sentido a la base de datos y en el mismo sentido las inconsitencias y la diversidad en las soluciones a dichos problemas se vuelven otro problema que resolver haciendo de la evolucion un proceso lento y lleno de parches y errores.
        
  
  \item ¿Qu\'e problemas operativos aparecer\'ian s\'i no se logra la independencia f\'isica, especialmente en t\'erminos de almacenamiento y rendimiento?

        Por otro lado la independencio fisica nos da la capacidad de optimizar como se guardan y ejecutan nuestros indices, particiones, discos... sin tocar las apps o esa parte intangible de nuestros sistemas. Sin ella las apps dependen de detalles "fisicos"
        \begin{itemize}
            \item No puedes optimizar sin tocar codigo.
            \item El rendimiento es inestable.
            \item Escalado y la migracion de nuestros datos se vuelven dificiles.
        \end{itemize}

  
  \item Proporciona un ejemplo realista en el que un cambio f\'isico no afecte la l\'ogica del sistema, y otro donde un cambio l\'ogico pueda impactar m\'ultiples aplicaciones s\'i no se gestiona adecuadamente.


         \begin{itemize}
         
             \item Cambio fisico que no afecta a la logica.
               Una empresa de logística guarda todos los registros de entregas en una tabla muy grande llamada entregas. Con el tiempo, esa tabla crece demasiado y las consultas se vuelven lentas. En este punto se busca mejorar el rendimiento, el administrador decide mover la tabla a un servidor más rápido con discos SSD y crear un índice en la columna $"fechae_ntrega"$.
                Para las aplicaciones dedicadas a las consultas de los datos, nada cambia: siguen haciendo preguntas como “dame todas las entregas de ayer” con el mismo comando SQL. Lo único que mejora es la velocidad y eficiencia. Es decir, por debajo cambia el almacenamiento, pero la lógica de cómo se consulta permanece igual.
                
             \item Cambio logico con multiples implicaciones.
                Tenemos una tabla clientes tiene un campo $tipo_cliente$ con dos valores: “regular” y “premium”. La empresa ahora crea un nuevo segmento de clientes corporativos y decide cambiar la estructura de la tabla: elimina la columna $tipo_cliente$ y crea una tabla nueva llamada $segmentos_clientes$ con un catálogo más amplio.
                  Si no se llevan estos cambios con cuidado las aplicaciones que dependían de $tipo_cliente$ dejarán de funcionar: reportes de ventas, facturación o hasta el sistema de seguimiento que valida beneficios de clientes especiales podrían fallar porque ya no encuentran la columna que usaban antes de los cambios.
         \end{itemize}

\end{itemize}




\item Describe el papel que tienen los Sistemas Manejadores de Bases de Datos (SMBD) en el enfoque de bases de datos. ¿Por qu\'e consideras que es importante (o no) que un administrador de bases de datos (DBA) conozca las caracter\'isticas de un SMBD?

    Es importante que un administrador de bases de datos (DBA) conozca a fondo las características de un SMBD porque su trabajo no solo es guardar datos, sino darnos la certeza que la base funcione de manera confiable, eficiente y segura. Por ejemplo, un DBA debe entender cómo configurar esquemas, gestionar usuarios y permisos, aplicar mecanismos de respaldo y recuperación, optimizar consultas e índices, y controlar el acceso concurrente del equipo para evitar bloqueos o pérdidas de información. De la misma forma, el conocimiento de laa caracteristicas particulares de cada SMBD (ya sea Oracle, MySQL, PostgreSQL, SQL Server, MongoDB, etc.) le permite elegir la mejor estrategia de mantenimiento, escalabilidad y seguridad de acuerdo con las necesidades de la de la organizacion y del proyecto.




\item Indica las responsabilidades que tiene un Sistema Manejador de Bases de Datos y para cada responsabilidad, explica los problemas que surgir\'ian si dicha responsabilidad no se cumpliera.
\textbf{Respuesta:}\\
Las principales responsabilidades de un \textbf{Sistema Manejador de Bases de Datos} son las siguientes:\\
\textbf{Definición:} Se podra decidir como estará organizada la Base de Datos para tener un orden de nuestros datos, su tipo, su estructura, relaciónes entre datos, restricciones que habrá y tener consistencia.

\hspace{0.3cm}\textbf{Problemas:} Si no se define, es de esperar encontrar diferentes problemas al momento en nuestra Base de Datos desde tipos incompatibles, datos duplicados, una estructura que no nos favorece, inconsistencias, etc.
    
    
\textbf{Construcción:}
Es necesario tener y llevar un control sobre los datos que estarán almacenados en la Base de Datos, pues siempre se buscará tener un orden y una organización de los datos, a su vez, también llevar una gestión del espacio en memoria/disco.

\hspace{0.3cm}\textbf{Problemas:} 
Al no contar con una construcción llevada a cabo por el SMBD, implicaría que los datos pudieran estar desordenados, desornanizados si no se almacenan de forma correcta, ademas sería dificil manejar grandes cantidades de datos por las incosistencias generadas. 

    
\textbf{Manipular:}
Es característica fundamental de un SMBD, pues podemos hacer operaciones con los datos almacenados desde realizar consultas, insertar nuevos datos, modificar, eliminar, entre otras más especificas. Nos facilita manipular los datos de una base de datos. 

\hspace{0.3cm}\textbf{Problemas:} 
Si no estuviera se cumpliera esto, resultaria en que no podriamos realizar operaciones básicas con nuestros datos como recuperar información, tal vez se tendría que programar dichas operaciones pero conlleva a tiempo invertido.

    
\textbf{Compartir:}
Es importante que más de un usuario pueda hacer uso de la Base de Datos al mismo tiempo (concurrencia), pues permite que todos los usuario que todos conectados vean en tiempo real lo que hay almacenado y asi no perder la integridad de los datos.

\hspace{0.3cm}\textbf{Problemas:} 
Si no fuera posible la concurrencia resultaría en la inconsistencia de los datos, por ejemplo: dos usuarios podrían manipular el mismo dato a la vez y se sobreescribirían los cambios realizados por cada uno. 
    
\textbf{Seguridad e Integridad:}
Debe garantizar seguridad a la Base de Datos para evitar el acceso de usuario no autorizados a los datos y así evitar que modifiquen o borren datos importantes. Ademas cada usuario puede ser asignado con ciertos permisos y limitar su control con la base de datos.
 
\hspace{0.3cm}\textbf{Problemas:} 
Si careciera de esto básico, entonces estarían en peligro todos los datos almacenados estando expuestos a modificaciones o eliminaciones por parte de cualquier persona/usuario que tuviera conexión a la Base de Datos.

\textbf{Recuperación;}
Es posible que el SMBD cuente con la opción de restaurar o recuperar los datos de la Base de Datos en caso de que se hubiera presentado un problema mayor y resulte en la perdida de los datos. Es recomendado siempre tener un respaldo de todos los datos para evitar cualquier percance y tener un plan de respaldo.

\hspace{0.3cm}\textbf{Problemas:}
Si no existiera está función de recuperación, es posible que cuando se presente un problema relacionado a los datos, ya sea en la modificación de los datos, migración o incluso en la perdida total, no sería posible recuperar dichos datos almacenados que pudieran ser importantes.


\item Investiga qu\'e es la \textit{redundancia de datos}. ¿Cu\'al ser\'ia la diferencia entre \textit{redundancia de datos controlada} y \textit{no controlada}? Proporciona un ejemplo claro de cada tipo. ¿Por qu\'e la redundancia no controlada suele ser un problema para la integridad y consistencia de los datos?
La \textbf{redundancia de datos} sucede cuando se almacena multiples veces un mismo dato en una Base de Datos, en varias ubicaciones o tablas.\\
Existen dos tipos de redundancia: la Redundancia Intencional (\textbf{Controlada}) y la Redundancia Involuntaria (\textbf{No Controlada}).\\
\textbf{Redundancia Controlada:} Sucede cuando la redundancia de datos es hecho con planificación previa, es planificada para poder sacarle provecho bajo una situacion controlada, puede mejorar el rendimiento en algunas ocasiones.
 
 \hspace{0.3cm}\textbf{Ejemplo:} Puede ser el caso en que un mismo dato este almacenado en distintas lugares (tablas, servidores) lo cual podra garantizar su disponibilidad  en el caso en que uno de los servidores fallé, y también protegerlo de pérdidas de datos.

\textbf{Redundancia No Controlada:} Ocurre cuando la redundacia es producto de una mala organizacion o un mal diseño en la Base de Datos, la duplicidad de un dato no fue deseado o planeado y puede generar ineficiencia en el almacenamiento.

 \hspace{0.3cm}\textbf{Ejemplo:} Pudiera ser el caso en donde tengamos a una misma persona registrada dos veces en la Base de Datos, puede que difieran en un campo/dato como el correo pero sigue siendo la misma persona y, por ende, guardando todos los demas datos dos veces (nombre, id, direccion, etc), lo que puede generar más costos no deseados en el almacenamiento si esta situacion sudece más veces.

 Ahora bien, la \textbf{redundancia No Controlada} puede generar inconsistencia y ser negativo para la integridad de los datos puesto que, como comentabamos antes en el ejemplo, puede tenerse más de una vez un mismo dato lo cual podría llevar a que al momento de querer modificarlo tiene que sincronizarse con los demas lugares en los que aparece pues si no se hace entonces perderiamos la consistencia en los datos y también perderiamos la integridad puesto que el dato puede llegar a tener distintos valores. Por esto mismo suele haber problemas cuando se trata de redundancia no controlada, pues viene consigo el degradamiento de rendimiento y aumente los costos con respecto al almacenamiento en la Base de Datos.


\item Imagina que formas parte del equipo de tecnolog\'ia de una empresa aseguradora que gestiona informaci\'on sobre p\'olizas, clientes, siniestros, evaluaciones de riesgo y cumplimiento normativo. Responde lo siguiente:
\begin{itemize}
  \item ¿Cu\'ales son las principales responsabilidades del DBA en este contexto, considerando la necesidad de confidencialidad, trazabilidad y disponibilidad continua de la informaci\'on?
  \item ¿Qu\'e habilidades t\'ecnicas y conocimientos espec\'ificos (por ejemplo, manejo de datos sensibles, normativas como protecci\'on de datos personales, respaldo y recuperaci\'on ante desastres) son m\'as cr\'iticos para este entorno?
  \item Si el DBA no ejecuta consultas directamente, ¿por qu\'e sigue siendo necesario que comprenda el modelo l\'ogico de la base de datos? Justifica tu respuesta considerando la importancia de la integridad de los datos en procesos como la evaluaci\'on de riesgos o la detecci\'on de fraudes.
\end{itemize}

\item A lo largo del tiempo han existido diversos \textit{modelos de datos}, como el modelo jer\'arquico, el modelo de red, el modelo relacional, el modelo orientado a objetos y los modelos NoSQL (llave--valor, documento, columnares, grafos). Responde lo siguiente:
\begin{itemize}
  \item Elabora una \textbf{tabla comparativa} entre al menos cuatro modelos de datos, destacando sus principales caracter\'isticas, ventajas, desventajas y casos de uso t\'ipicos.
  \item ¿Por qu\'e crees que el modelo relacional se convirti\'o en el m\'as ampliamente adoptado durante d\'ecadas?
  \item ¿En qu\'e contextos actuales, los modelos NoSQL superan al modelo relacional en eficiencia o flexibilidad? Justifica con ejemplos.
\end{itemize}

\item Sup\'on que deseas crear una aplicaci\'on para \textit{gesti\'on hospitalaria}. Considera cada una de las desventajas indicadas en el documento \textit{``Purpose of Database Systems''}, cuando se administran los datos en un sistema de archivos. Discute la relevancia de cada uno de los puntos indicados con respecto a la gesti\'on de datos de pacientes: historial m\'edico, diagn\'osticos, tratamientos, citas, acceso a registros m\'edicos, m\'edicos, especialidades, entre otros.
\end{enumerate}
\newpage
\item \textbf{Lectura de art\'iculo}

\begin{enumerate}[label=\textbf{\alph*.}, leftmargin=*, itemsep=1.0em]

\item Leer el art\'iculo \textit{The value of our personal data in the Big Data and the Internet of all Things Era} y realizar un \textbf{resumen} del documento, destacando los puntos que a su consideraci\'on sean los m\'as relevantes (no m\'as de dos cuartillas).
\begin{itemize}
  \item Redacten en sus propias palabras, eviten copiar frases textuales del art\'iculo.
  \item Contenido m\'inimo: tema central del art\'iculo, ideas principales que desarrolla el autor, conclusi\'on o aportaci\'on m\'as importante del texto.
  \item En cuanto al formato, procuren tener p\'arrafos bien estructurados, \textbf{sin vi\~netas ni listas}.
\end{itemize}

\textbf{Resumen}

%-----parte aron 
En este artículo se analiza cómo la cuarta revolución industrial afecto el mundo como lo conocíamos, impulsada por la tecnología digital, las redes sociales y los dispositivos conectados. Por ello ha transformado nuestra manera de interactuar con el mundo y ha convertido la información personal en un recurso altamente valioso. Y debido a esto las empresas tecnológicas aprovecharon y aprovechan los datos generados por los usuarios para obtener ventajas competitivas, monetizando la información y transformando nuestra identidad digital en un activo económico.

Ademas, muchas veces los usuarios, desconocen qué datos están compartiendo y con quien lo comparten. Este intercambio constante de información a cambio de servicios gratuitos pone en riesgo la privacidad y la confianza en el mercado digital. 
Por ello nuestra identidad se redefine por nuestro valor en información digital. Por ejemplo, los servicios de redes sociales, como Facebook y Google, no venden productos a los usuarios, sino que los mismos usuarios son activos para anunciantes, transformando nuestra información, ideas y hábitos en mercancía.

Ya que nosotros como usuarios no comprendemos el uso que se le puede dar a nuestros datos en aquel mercado digital y por ello nuestra privacidad, la propiedad de la información y los derechos humanos se ven comprometidos, y surge la pregunta de cómo medir la confianza y el valor de nuestros datos si no somos conscientes de su importancia en la economía digital.

%----parte Santiago págs. 76 y 77 
Igualmente, en esta última parte del texto se analiza la importancia de la protección de datos personales en el contexto del crecimiento del Big Data y el Internet de las Cosas, y cómo iniciativas como la del proyecto "Hub of All Things" buscan regresar al usuario el control sobre su propia información, operando como una plataforma digital donde los datos recolectados de las distintas formas en el día a día del usuario son almacenados y gestionados por los propios usuarios, quienes tienen las decisiones sobre con quién se comparten sus datos y para qué.

En este sentido, también se habla de la intervención de la Unión Europea y su postura ante esta situación, enfatizando que la confianza de los usuarios es esencial para el desarrollo económico y ha establecido el Reglamento General de Protección de Datos (GDPR) como un marco regulatorio único que regula el tratamiento de datos personales. Ello tomando en cuenta que los datos personales se han convertido en un activo económico.

En la recta final, al llegar a la parte de la conclusión, el documento enfatiza que los datos no pueden ser considerados neutrales, ya que reflejan cómo nos relacionamos con nuestro entorno. Señalando la importancia de que las personas sean consistentes con el manejo de sus datos e información, promoviendo la creación de Personal Data Stores (PDS) como vía para empoderar a las personas y fomentar un entorno más justo, donde el valor de los datos no quede concentrado únicamente en las empresas y organizaciones, sino que de la misma forma tenga la capacidad de beneficiar al individuo.






\item Realizar un \textbf{ensayo} donde expresen sus comentarios (cada integrante del equipo deber\'a indicar este punto de forma individual en el documento que redacten) sobre la lectura, considerando los siguientes puntos:

\begin{itemize}
  \item \textbf{Extensi\'on:} entre 1 y 2 cuartillas.
  \item \textbf{Estructura obligatoria:}
  \begin{itemize}
    \item \textbf{Introducci\'on:}
    \begin{itemize}
      \item Presenta brevemente el tema del art\'iculo.
      \item Explica por qu\'e consideras que es importante o relevante.
      \item Plantea tu postura o idea principal.
    \end{itemize}
    \item \textbf{Desarrollo:}
    \begin{itemize}
      \item Resume de forma breve los argumentos del autor.
      \item Relaciona las ideas del art\'iculo con otros conocimientos, lecturas o ejemplos.
      \item Exp\'on tus propios argumentos (a favor o en contra), justificando con razones claras.
      \item Puedes incluir ejemplos de la vida real, casos de estudio o tu experiencia personal.
      \item Deber\'as indicar cu\'al es el objetivo que quiso plantear el autor: qu\'e intenta decir, de qu\'e intenta persuadirnos y/o convencernos, ¿c\'omo se relaciona con la materia de \textit{Fundamentos de Bases de Datos}?
      \item Deber\'as indicar cu\'al es la tem\'atica central del art\'iculo y se debes se\~nalar el tema o los temas laterales que desarrolla el mismo y c\'omo estos tienen relaci\'on con t\'u pr\'actica profesional.
    \end{itemize}
    \item \textbf{Conclusi\'on:}
    \begin{itemize}
      \item Reafirma tu postura inicial.
      \item Explica qu\'e aprendiste o qu\'e aportaci\'on consideras m\'as valiosa.
      \item Se\~nala si el tema abre preguntas o retos futuros.
    \end{itemize}
    \item \textbf{Formato:}
    \begin{itemize}
      \item Redacci\'on en p\'arrafos completos, sin listas.
      \item Usa conectores l\'ogicos (por lo tanto, adem\'as, sin embargo, en conclusi\'on, etc.).
      \item \textbf{Ortograf\'ia y gram\'atica cuidadas}.
    \end{itemize}
  \end{itemize}
\end{itemize}



\newpage
\textbf{Ensayo\\
Autor: Santiago Zapata Amezcua}

Hoy en día, tras la evolución de la tecnología y la llegada de la cuarta revolución industrial, los datos e información personal se han convertido en un recurso codiciado y estratégico para la economía. Es así que cada interacción que realizamos en plataformas digitales deja huellas e información que, sin parecer valiosa, es el pilar de la toma de decisiones en las empresas. Siendo una situación de gran importancia, ya que plantea un gran dilema sobre la privacidad, los derechos humanos y el papel de nosotros, las personas, como generadores de riqueza para las grandes empresas. Y si bien los datos representan una fuente de innovación y desarrollo, es indispensable que nosotros como usuarios juguemos un papel más importante en el manejo de nuestra propia información para empezar a frenar el constante crecimiento del control que las empresas tienen sobre nosotros.\\

El Big Data y el Internet de las Cosas han multiplicado la generación y, sobre todo, la utilidad de la información al grado que hoy en día se habla del "tsunami de datos" como un fenómeno que define nuestros tiempos. Como individuos hemos pasado a convertirnos en activos intangibles, donde nuestra información personal es una moneda de cambio para acceder a servicios presentados como gratuitos, pero donde en realidad lo que dejamos de forma gratuita es nuestra información.\\

Y es aquí donde se plantea el verdadero dilema, ya que al día de hoy la falta de conciencia sobre el gran valor de nuestra información y nuestro bajo o nulo control sobre ella en el mundo digital. De la misma forma, es importante destacar la labor de los organismos gubernamentales que, ante dichos dilemas, han comenzado a plantear y ejecutar estrategias para dar justicia en la relación y acceso a la información. Un ejemplo es la Unión Europea, organismo que mediante el Reglamento General de Protección de Datos (GDPR) busca establecer un marco que garantice derechos, lo que refleja el comienzo del esfuerzo institucional por equilibrar la balanza entre las empresas y los usuarios.\\

En este sentido, es clave entender que los datos personales funcionan como activos económicos, llegando a ser comparables con recursos naturales como el petróleo por su gran valor de cambio y su capacidad para generar riqueza a partir de ellos. Situación que se relaciono directamente con la materia de FBDD, ya que uno de los pilares de esta es comprender cómo se almacenan, estructuran y protegen los datos para garantizar integridad y seguridad. Un buen diseño de sistemas debe incluir mecanismos que respeten la privacidad y a la vez permitan aprovechar la información de forma ética.\\

En conclusión, es importante reafirmar que para la plena justicia en esta era digital es vital el equilibrio en el control y acceso a la información, sobre todo evitando dejar caer todos los recursos sobre las empresas, como sucede hoy en día, pues ello solo alargará más la brecha de injusticia en el poder de la información. El aprendizaje más valioso que transmite la lectura es que es indispensable empoderar a los individuos a través de iniciativas gubernamentales, educativas y empresariales para retomar el acceso al control y manejo de su información. Y si bien el tema abre horizontes a nuevos retos a futuro sobre cómo equilibrar innovación y derechos fundamentales, cómo garantizar justicia en el uso de las nuevas ciencias y tecnologías y, sobre todo, cómo construir una cultura digital más consciente y activa sobre uno de los activos más importantes en la economía contemporánea: la información.

        









\newpage
\textbf{Ensayo\\
        Autor: Aar\'on L\'opez Mendoza}   


En la actualidad gracias al Big Data y el Internet de las Cosas (IoT) nos dan cosas increíbles, como servicios súper personalizados y maneras más fáciles de hacer trámites, sin embargo también es verdad que hay un lado oscuro que no podemos ignorar.\\
Piénsalo de esta manera: cada vez que tu o una persona le da "aceptar" a los términos y condiciones, abre una app o usan su celular para cualquier acción que involucre una red, están dejando un rastro de datos. Esos datos son tuyos, son como tu huella digital, o tu CURP y con ellos, las empresas, gobiernos o cualquier tercero con conocimientos pueden entenderte mejor, predecir tus gustos y hasta influir en tus decisiones.
Por ello el problema es que, mientras generamos información a una velocidad indescriptible, la seguridad que debería proteger esta información parece ir a paso de tortuga. Y esto nos deja vulnerables. Terminamos expuestos a que alguien robe nuestra identidad, que haya filtraciones masivas que expongan nuestra vida privada, o que nuestra información sea usada para manipularnos, ya sea para vendernos algo o para influir en nuestra forma de pensar.\\

Y a menudo, los permisos que aceptamos son tan enredados y confusos que ni siquiera sabemos a qué estamos diciendo que sí, sin embargo siempre aceptamos, ya que es muy seguro que 2 de cada 10 personas se detienen a leer los términos y condiciones completos, ya que asumimos que nuestros datos estarán seguros, a pesar de que haya una alta cantidad de datos en circuito o uso. Por ello la falta de transparencia es un problema enorme, que hasta la fecha seguimos sin buscar una solución.
Entonces, ¿el Big Data y el IoT son buenos o malos? La verdad es que, en este momento, los riesgos superan a los beneficios. Mientras no haya reglas claras, herramientas de protección que realmente funcionen y una cultura digital que ponga a las personas en primer lugar, nuestra privacidad y seguridad personal están en riesgo. La tecnología puede ser una herramienta poderosa y muy útil, pero también puede ser una amenaza si no la manejamos con cuidado.\\

Además, es importante reconocer que el uso de aplicaciones e infomaci\'on digital juega un papel fundamental en todo esto. No basta solo con confiar en que las empresas o los gobiernos protegerán nuestros datos; nuestra responsabilidad como usuarios es que debemos aprender a identificar riesgos, configurar correctamente la privacidad de nuestras cuentas y ser conscientes de qué información compartimos y con quién. El conocimiento en el ámbito digital no solo nos ayuda a protegernos, sino que también nos proporciona una gran habilidad para tomar decisiones informadas sobre nuestra interacción con la tecnología, permitiéndonos aprovechar sus beneficios sin poner en riesgo nuestra seguridad y nuestra vida privada. Por lo que debemos preguntarnos a partir de que tomemos la desici\'on de ingresar al mundo digital, que es lo que queremos hacer y a quienes queremos mostrarle lo que hacemos, con eso en mente podremos tomar mejores decisiones en el mundo digital.


\newpage
\textbf{Ensayo\\
        Autor: Isaac Giovani Escobar Gonzalez}

Actualmente nuestros datos personales han adquirido un valor en el mercado digital con el pasar de los años y el gran constante avance de las tecnologías recientes, estó sin que la mayoria de nosostros los consumidores nos hayamos dado cuenta. El artículo \textit{The value of our personal data in the Big Data and the Internet of all Things Era} aborda principalmente sobre el uso de nuestros datos por parte de las grandes empresas y tecnologías como \textbf{Big Data} o \textbf{IoT} convierten nuestra información en algo valioso para las compañias que pagarían por tener acceso a multiples datos. Esto es muy preocupante puesto que no sabemos que limitantes o restricciones estamos poniendo sobre nuestros datos compartidos y lo que las grandes empresas pueden hacer con ellos, es importante tener conocimiento y conciencia sobre el uso que se le puede llegar a tomar nuestros datos a nosotros como personas digitales, debemos saber si pueden llegar a ser beneficiosos, o bien, que lleguemos a obtener alguna forma de recompensa por el valor que pueden tomar nuestros 'activos' (datos) y empezar a limitar, obtener completo control sobre nuestros datos confidenciales u otros y tener derecho a nuestra propia privacidad digital.\\

Se hace mención el como existe una 'asimetría' de poder entre los usuario y las grandes empresas, existe un desconocimiento por parte de los usuarios lo que pueden llegar a valer su información/datos proporcionada, lo cual puede ser aprovechado por las empresas y tomar control de dichos datos y puede plantearse una violación hacia la privacidad y llegar a perder la confianza en el mercado. Recientemente esto ha sido motivo para que organizacionesa a favor del usuario desarrollen mecanismos para darle más control y poder al usuario y pueda seguir equilibrando la balanza, estos mecanismos fueron implementados para hacer sentir más seguro al usuario, tener derechos, todo esto lo cual considero que es un gran avance acertado para el beneficio de los miles de millones de usuarios que constantemente se encuentran compartiendo datos, pues con el hecho de tan solo interartuar con un dispositivo teconologíco, el internet, reseña, experiencia, etc, es más que sufiente para las corporaciones quieran conseguir nuestras interacción para sacarle provecho.\\

En mi opinión, es necesario que más personas, por cuenta propia, sean concientes del valor que pueden tomar sus datos, ya que hoy en día (y estoy a favor de lo que menciona el autor) nosotros como personas hemos empezado a ser definidos por nuestra '\textbf{huella digital}' que por lo físico, puesto que ahora se nos puede considerar solamente con 0's y 1's por la gran interacción que tenemos con el internet. Muchas personas comparten la idea de que estos servicios son gratis y que no pagamos nada, pero la realidad es otra, el simple hecho de subir fotos, mostrar nuestros gustos, dar likes o inclusive usar la inteligencia artificial, nos hace que formemos parte del Big Data, Nuestros datos pueden ser recopilados y usados, por ejemplo: para mostrarnos contenidos o anuncios plublicitarios sobre nuestros gustos o contenido que consumamos. Bajo mi \textbf{propia experiencia} puedo decir que me ha pasado muchas veces que cuando estoy navegando por redes sociales (Facebook, Instagram, Tiktok) empiezo a recibir contenido relacioado a lo que consumo en mi inicio, incluso cuando converso con compañeros/amigos sobre un tema o algo, al poco tiempo empiezo a notar como me sucede lo mismo, noto como mis redes sociales me llenan de contenido referente a lo hablado, lo cual me parece muy curioso y a veces gracioso, pero me deja ver como todo con lo que interactuamos puede influir en nuestro entorno digital. \\

Notemos como todo lo abordado en el artículo acerca del \textbf{Big Data} y \textbf{IoT} implica que se haga el uso de las Bases de Datos puesto que son capaces de poder almacenar grande volúmenes de datos y ofrecen una velocidad y eficiencia la cual siempre sera requerida para poder manejar grandes cantidades de datos y poder ser convertidos en información valiosa para mulitples objetivos para las empresas, no podría ser posible si se usaran otros servicios como lo puede ser Excel puesto que tienen sus limitante, por eso hace sentido contar las características de una Base de Datos y su seguridad e integridad que pueden llegar a ofrecer. \\

Para terminar, quiero comentar que los datos que siempre publiquemos o publiquemos en servicios web o de internet siempre tendran valor de alguna u otra manera y será importante que cada persona sea conciente de esto. Debemos saber en que lugares compartimos nuestra huella digital y que cada uno deberiamos tener un control sobre los mismos. A partir de este artículo pude conocer el valor significante que pueden tomar mis datos para algunas empresas y el como pueden formar parte del Big Data y IoT que son conceptos nuevos para mi. Personalmente no soy alguien que publique muchas cosas o dé información de mi vida en redes sociales, pero puedo ver lo mucho que pueden influir dichos datos. Creo que a futuro, esté es un tema que seguira dando de que hablar, con el tiempo las organizaciones y leyes establecidas equilibrarán la balanza entre usuario y compáñias con respecto a los datos, sobre todo para los usuarios puesto que tendran control de sus datos y también de su vida digital, para que finalmente los usuarios puedan ser dueños de sus propios datos. 
\end{enumerate}

\end{enumerate}


\begin{thebibliography}{99}
    \bibitem{Yung}
    Yung, Z. (2025, 5 agosto). Spreadsheet vs. Database: Honest Answers to Your Frequent Questions. Coupler.io Blog. \url{https://blog-coupler-io.translate.goog/spreadsheet-vs-database/?_x_tr_sl=en&_x_tr_tl=es&_x_tr_hl=es&_x_tr_pto=tc}

    \bibitem{Ganchev}
    Ganchev, M. (2022, 3 mayo). Database vs Spreadsheet: What is the Difference? 365 Data Science. \url{https://365datascience-com.translate.goog/tutorials/sql-tutorials/database-vs-spreadsheet/?_x_tr_sl=en&_x_tr_tl=es&_x_tr_hl=es&_x_tr_pto=tc&_x_tr_hist=true}

    \bibitem{Biuwer}
    Por qué no debes utilizar Excel como Base de Datos | Biuwer. (s. f.). \url{https://biuwer.com/es/blog/por-que-no-debes-utilizar-excel-como-base-de-datos}

    \bibitem{Oracle}
    Oracle. (s. f.). Introduction to Oracle Database. https://docs.oracle.com/en/database/oracle/oracle-database/23/cncpt/introduction-to-oracle-database.html

    \bibitem{IBM}
    IBM. (n.d.). Database management systems (DBMSs). https://www.ibm.com/think/topics/database
    
    \bibitem{IBM}
    Krantz, T., Jonker, A. (2025, 7 febrero). Redundancia de datos. IBM. \url{https://www.ibm.com/mx-es/think/topics/data-redundancy}

    \bibitem{DB book}
    Silberschatz, A., Korth, H. F., & Sudarshan, S. (2019). Database system concepts (7.ª ed.). McGraw-Hill. https://www.db-book.com/
    
    \bibitem{celerdata}
    CelerData. (2024, 5 octubre). Data Redundancy: What It Is and How to Manage It. CelerData. \url{https://celerdata.com/glossary/data-redundancy}

    \bibitem{IONOS}
    SGBD: introducción al sistema gestor de base de datos. (2020, 12 mayo). IONOS Digital Guide. \url{https://www.ionos.mx/digitalguide/hosting/cuestiones-tecnicas/sistema-gestor-de-base-de-datos-sgbd/}
\end{thebibliography}
\end{document}
